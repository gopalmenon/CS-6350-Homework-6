\section{Warmup: Probabilities}

For the following questions, suppose $A_1, A_2, A_3, A_4$ are
events.\\\textit{(Remember that no points will be awarded without explanations.)}

\begin{enumerate}
\item \relax[2 points] If $P(A_1) = P(A_2) = P(A_1 \mid A_2) = \frac{1}{2}$, then
  are the events $A_1$ and $A_2$ independent? Why?
  
Events $A_1$ and $A_2$ are independent if 
  
$$
P(A_1\mid A_2) = P(A_1)\\
$$
and
$$
P(A_2\mid A_1) = P(A_2)\\
$$

Using Bayes rule
\begin{equation*}
\begin{aligned}
P(A_2\mid A_1) &= \frac{P(A_1\mid A_2)P(A_2)}{P(A_1)} \\
&= \frac{\frac{1}{2}\times \frac{1}{2}}{\frac{1}{2}}\\
&= \frac{1}{2}\\
&=P(A_2)\\
&= P(A_1 \mid A_2)\\
&= P(A_1)
\end{aligned}
\end{equation*}
  
\item \relax[3 points] Suppose $A_1, A_2$ and $A_3$ are mutually
  exclusive. If, for $i \in \{1,2,3\}$, we have $P(A_i) = \frac{1}{3}$
  and $P(A_4 \mid A_i) = \frac{i}{6}$, then what is $P(A_4)$?
  
Using the theorem of total probability in the above case

\begin{equation*}
\begin{aligned}
P(A_4) &= \sum_{i=1}^3 P(A_4 \mid A_i) P(A_i)\\
&= \sum_{i=1}^3 \frac{i}{6} \times \frac{1}{3}\\
&= \frac{1}{3} \times \left ( \frac{1}{6} + \frac{2}{6} + \frac{3}{6} \right )\\
&= \frac{1}{3} \times \frac{6}{6}\\
&= \frac{1}{3} 
\end{aligned}
\end{equation*} 
 
\item \relax[3 points] Let $n$ be the number at the top when a fair
  six-sided die is tossed. If a fair coin is tossed $n$ times, then
  what is the probability of exactly two heads?

Let $H$ be the event of getting a head, $2H$ be the event of getting exactly two heads when tossing a coin, and let $D_n$ be the event for the number at the top when a die is tossed.

So the probability of exactly two heads is
\begin{equation*}
\begin{aligned}
&= \sum_{n=1}^6  P(2H \mid D_n) \times P(D_n)\\
&= \sum_{n=1}^6  P(2H \mid D_n) \times \frac{1}{6}\\
&= \sum_{n=1}^6  {n \choose 2} \times \left( P(H) \right ) ^ 2  \times \left(1- P(H) \right ) ^ {n-2} \times \frac{1}{6}\\
&= \frac{1}{6} \times \bigl [ 0 +  1 \times \left(\frac{1}{2} \right)^2 \times \left (\frac{1}{2}\right )^0 + 3 \times \left(\frac{1}{2} \right)^2 \times \left (\frac{1}{2}\right )^1 + 6 \times \left(\frac{1}{2} \right)^2 \times \left (\frac{1}{2}\right )^2 \\
&+10 \times \left(\frac{1}{2} \right)^2 \times \left (\frac{1}{2}\right )^3  + 15 \times \left(\frac{1}{2} \right)^2 \times \left (\frac{1}{2}\right )^4 \bigr ]\\
&= \frac{1}{6} \times \bigl [ 0 +  \frac{1}{4} + \frac{3}{8} + \frac{6}{16} + \frac{10}{32} + \frac{15}{64}\bigr ]\\
&= \frac{1}{6} \times \bigl [ \frac{16+24+ 24+20+15}{64}\bigr ]\\
&=\frac{1}{6} \times\frac{99}{64} \\
&=\frac{33}{128}
\end{aligned}   
\end{equation*}  
  
\item \relax[4 points] Prove or disprove: If $P(A_1) = a_1$ and
  $P(A_2) = a_2$, then $P(A_1 | A_2) \geq \frac{a_1 + a_2 -1}{a_2}$.
\item \relax[8 points] If $A_1$ and $A_2$ are independent events, then
  show that
  \begin{enumerate}
  \item $E[A_1 + A_2] = E[A_1] + E[A_2]$

The expected value of a random variable $X$ is defined as 

\begin{equation*}
\begin{aligned}
E(X) = \sum_{e \in S} X(e) P(e)
\end{aligned}
\end{equation*} 

where $e$ is an elementary event in probability space $S$.
\end{enumerate}


\begin{equation*}
\begin{aligned}
E(A_1 + A_2) &= \sum_{e \in S} \left \{A_1(e) + A_2(e) \right \} P(e)\\
&= \sum_{e \in S} A_1(e) P(e)+ A_2(e) P(e)\\
&= E(A_1)  + E(A_2) 
\end{aligned}
\end{equation*}

  \item $var[A_1 + A_2] = var[A_1] + var[A_2]$
  
The variance of a random variable $X$ is defined as 

\begin{equation*}
\begin{aligned}
var(X) &= E((X-E(X))^2)
\end{aligned}
\end{equation*} 

based on this definition

\begin{equation*}
\begin{aligned}
var(A_1 + A_2) &= E((X-E(A_1 + A_2))^2)\\
&= E((X-(E(A_1) +E( A_2))^2)
\end{aligned}
\end{equation*} 

  \end{enumerate}

  Here $E[\cdot]$ and $var[\cdot]$ denote the mean and variance
  respectively.
%%% Local Variables:
%%% mode: latex
%%% TeX-master: "hw6"
%%% End:
