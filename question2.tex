\section{Naive Bayes}

\newcommand{\ph}{\hat{P}}

In this question, we will examine the influence of the conditional
independence assumption that the naive Bayes model makes. For this
question, we will use the notation $P$ to denote the true
probabilities that generates the data and $\ph$ to denote
probabilities that are learned from data. Assume no smoothing is done.



\begin{enumerate}
\item \relax[Part 1] Suppose we have a binary classification problem
  where the label $y$ can either be $-1$ or $1$. In the first case,
  consider the case where we have only one feature $x_1$ that can also
  be either $-1$ or $1$. The generative distribution of the data is
  $P(x_1, y) = P(y) P(x_1 \mid y)$. Note that this satisfies the
  independence assumption of the naive Bayes model. All features are
  conditionally independent of each other given the label -- of
  course, there is only one feature so this statement is trivially
  true.

  Suppose we know the true distribution that generated the data as
  follows:

  \begin{itemize}
  \item $P(y = -1) = 0.1$ and $P(y = 1) = 0.9$
  \item $P(x_1 = -1 \mid y = -1) = 0.8$, $P(x_1 = 1 \mid y = -1) = 0.2$, $P(x_1 = -1 \mid y = 1) = 0.1$ and $P(x_1 = 1 \mid y = 1) = 0.9$.
  \end{itemize}

  \begin{enumerate}
  \item \relax[2 points] If we have infinite data drawn from this
    distribution and we train a naive Bayes classifier, what would the
    values of $\ph(x_1 \mid y)$ and $\ph(y)$ be?
    
According to \textit{Hoeffdings Inequality} \cite{AMLBook}, the probability distribution of a random variable $\nu$ will be very close to its mean value $\mu$ for large samples. For a sample size of $N$, 

$$
P \left [ |\nu - \mu| > \epsilon \right ] \leq 2 e ^ {-2\epsilon^2N}
$$

for any $\epsilon > 0$.

When $N$ is $\infty$, the values of $\ph(x_1 \mid y)$ and $\ph(y)$ will be the same as $P(x_1 \mid y)$ and $P(y)$.
    
  \item \relax[6 points] Use these learned values probabilities from
    the previous question to fill up the following table:

    \begin{tabular}{|c|c|c|c|}
      \hline
      {\bf Input $x_1$} & $\ph(x_1, y= -1)$ & $\ph(x_1, y=1)$ & {\bf Prediction: $y' = arg\max_y \ph(x_1, y)$} \\
      \hline
      -1                & 0.8                   &   0.1              &  -1                                 \\
      1                 &  0.2                 &  0.9               &   1                                \\
      \hline
    \end{tabular}

  \item \relax[3 points] If the probabilities learned above were used to
    make predictions, what would the error of that classifier be? In
    other words, what is $P(y' \ne y)$? 

    Hint: To answer this, you should use the fact that
    $P(y' \ne y) = P(y' \ne y, x_1 = -1) + P(y' \ne y, x_1 = 1)$.

\begin{equation*}
\begin{aligned}
P(y' \ne y) &= P(y' \ne y, x_1 = -1) + P(y' \ne y, x_1 = 1)\\
P(y' \ne y) &= P(y=1, x_1 = -1) + P(y=-1, x_1 = 1)\\
P(y' \ne y) &= P( x_1 = -1, y=1) + P(x_1 = 1,y=-1)\\
P(y' \ne y) &= P( x_1 = -1 \mid y=1)P(y=1) + P(x_1 = 1 \mid y=-1)P(y=-1)\\
&= 0.1 \times 0.9 + 0.2 \times 0.1\\
&= 0.09+0.02\\
&=0.11
\end{aligned}
\end{equation*} 

  \end{enumerate}

\item \relax[Part 2] Now, suppose we have a binary classification
  problem with two features $x_1, x_2$ both of which can be $-1$ or
  $1$. However, the second feature $x_2$ is actually identical to the
  first feature $x_1$. And we have the same true probabilities
  $P(x_1 \mid y)$ and $P(y)$ as in Part 1 above.

  \begin{enumerate}
  \item \relax[1 point] Are $x_1$ and $x_2$ conditionally independent
    given $y$? Prove your answer formally using the definition of
    conditional independence.

Since the features $x_1$ and $x_2$ are identical,
    
\begin{equation*}
\begin{aligned}
P(x_1,x_2 \mid y) = P(x_1 \mid y) = P(x_2 \mid y)
\end{aligned}
\end{equation*} 

For $x_1$ and $x_2$ to be conditionally independent given $y$, the following should hold true

\begin{equation*}
\begin{aligned}
P(x_1,x_2 \mid y) = P(x_1 \mid y) P(x_2 \mid y)
\end{aligned}
\end{equation*} 

The only cases where the product of two probabilities is the same as the individual probabilities is when both are $0$ or when both are $1$. This means that the above two equations cannot be true for all cases of probability values and so $x_1$ and $x_2$ are not conditionally independent given $y$.
    
  \item \relax[8 points] Let $\ph(x_1 \mid y), \ph(x_2 \mid y)$ and
    $\ph(y)$ represent the learned parameters of a naive Bayes
    classifier that is learned on infinite data generated according to
    the above distribution. Using these parameters, fill up the
    following table:


    \begin{tabular}{|c|c|c|c|c|}
      \hline
      $x_1$ & $x_2$ & $\ph(x_1, x_2, y= -1)$ & $\ph(x_1, x_2, y=1)$ & {\bf Prediction: $y' = arg\max_y \ph(x_1, x_2, y)$} \\
      \hline
      -1    & -1    &    0.64                    & 0.01                     &    -1                                               \\
      -1    & 1     &     0.16                   &  0.09                    &     -1                                              \\
      1     & -1    &      0.16                  &   0.09                   &      -1                                             \\
      1     & 1     &      0.04                  &   0.81                   &       1                                            \\
      \hline
    \end{tabular}

  \item \relax[3 points] If the probabilities learned above were used to
    make predictions, what would the error of that classifier be? In
    other words, what is $P(y' \ne y)$?


\begin{equation*}
\begin{aligned}
P(y' \ne y) &= P(y' \ne y, x_1 = -1, x_2=-1) + P(y' \ne y, x_1 = -1, x_2=1) \\&+ P(y' \ne y, x_1 = 1, x_2=-1) + P(y' \ne y, x_1 = 1, x_2=1) \\
P(y' \ne y) &= P(x_1 = -1, x_2=-1, y' \ne y) + P(x_1 = -1, x_2=1,y' \ne y) \\&+ P(x_1 = 1, x_2=-1,y' \ne y) + P(x_1 = 1, x_2=1,y' \ne y) \\
P(y' \ne y) &= P(x_1 = -1, x_2=-1, y=1) + P(x_1 = -1, x_2=1,y=1) \\&+ P(x_1 = 1, x_2=-1,y=1) + P(x_1 = 1, x_2=1,y=-1) \\
P(y' \ne y) &= P(x_1 = -1, x_2=-1 \mid y=1) P(y=1)+ P(x_1 = -1, x_2=1 \mid y=1) P(y=1) \\&+ P(x_1 = 1, x_2=-1 \mid y=1) P(y=1) + P(x_1 = 1, x_2=1 \mid y=-1) P(y=-1) \\
&= 0.1 \times 0.9 + 0 \times 0.9 + 0 \times 0.9 + 0.2 \times 0.1\\
&= 0.09 + 0.02\\
&= 0.11
\end{aligned}
\end{equation*} 


  \item \relax[2 points] Do you expect a logistic regression
    classifier to have the same performance as the naive Bayes
    classifier when the variable is duplicated? Give an intuitive
    explanation (no more than 2 sentences) for your answer.

Given that both Na\"ive Bayes and Logistic Regression classifiers have a linear decision boundary, the decision boundary has to be the same. Since both classifiers will in effect learn the same linear decision boundary, they will predict the same output.
  \end{enumerate}

\end{enumerate}




%%% Local Variables:
%%% mode: latex
%%% TeX-master: "hw6"
%%% End:
