\section{[25 points, Extra Credit for the holidays] Na\"ive Bayes and Linear Classifiers }
\label{sec:q1}


In this problem you will show that a Gaussian na\"ive Bayes classifier is a linear classifier. We will denote inputs by $d$ dimensional vectors, $\bx = (x_1, x_2,\dots, x_d)^T$. We will assume that each feature $x_j$ is a real number. Our classifier will predict the label 1 if $\Pr(y=1|\bx) \geq \Pr(y=0|\bx)$. Or equivalently,$\frac{\Pr(\bx | y=1)\Pr(y=1)}{\Pr(\bx | y=0)\Pr(y=0)} \geq 1$. Remember the na\"ive Bayes assumption we saw in class: $\Pr(\bx|y) = \prod_{j=0}^{d}\Pr(x_j|y)$\\

Suppose each $P(x_j|y)$ is defined using a Gaussian/Normal probability density function, one for each value of $y$ and $j$. Each Gaussian distribution has mean $\mu_{j,y}$ and variance $\sigma^2$ (Note that they will all have same variance). As a reminder, the Gaussian distribution is represented by the following probability density function:$f(x_j\,|\,\mu_{j,y}, \sigma) = \frac{1}{\sqrt{2\sigma^2\pi}}e^{-\frac{(x_j-\mu_{j,y})^2}{2\sigma^2}}$\\

Show that this na\"ive Bayes classifier has a linear decision boundary.\\ \textit{[Hint: Refer to the notes on the naive Bayes classifier and Linear models in the class website to see how to do this with binary features]}\\

The classifier will predict a label of $1$ if 

\begin{equation*}
\begin{aligned}
P(y=1 \mid \bx) \geq P(y=0 \mid \bx)
\end{aligned}
\end{equation*}

or equivalently if

\begin{equation*}
\begin{aligned}
\frac{P(y=1 \mid \bx) }{ P(y=0 \mid \bx)} \geq 1\\
\frac{P(y=1) P(\bx  \mid y=1) }{ P(y=0) P(\bx \mid y=0)} \geq 1
\end{aligned}
\end{equation*}

Using the Na\"ive Bayes assumption

\begin{equation*}
\begin{aligned}
\frac{P(y=1) \displaystyle \prod_{j=0}^{d} P(\bx_j  \mid y=1) }{ P(y=0) \displaystyle \prod_{j=0}^{d} P(\bx_j \mid y=0)} \geq 1
\end{aligned}
\end{equation*}

As per the normal probability distribution assumption described above in the question, the classifier will predict a label of $1$ if,

\begin{equation*}
\begin{aligned}
\frac{P(y=1)}{P(y=0)} \displaystyle \prod_{j=0}^{d}  \frac{ \frac{1}{\sqrt{2 \sigma^2 \pi}} \exp\left(\frac{-\left(x_j - \mu_{1_{j,y}}\right)^2}{2\sigma^2}\right)}{   \frac{1}{\sqrt{2 \sigma^2 \pi}} \exp\left(\frac{-\left(x_j - \mu_{2_{j,y}}\right)^2}{2\sigma^2}\right)} &\geq 1\\
\frac{P(y=1)}{P(y=0)} \displaystyle \prod_{j=0}^{d} \frac{\exp\left(\frac{-\left(x_j - \mu_{1_{j,y}}\right)^2}{2\sigma^2}\right)}{\exp\left(\frac{-\left(x_j - \mu_{2_{j,y}}\right)^2}{2\sigma^2}\right)} &\geq 1\\
\ln \left( \frac{P(y=1)}{P(y=0)}\right) + \displaystyle \sum_{j=0}^{d} \left(  - \frac{\left( x_j - \mu_{1_{j,y}} \right)^2}{2\sigma^2} + \frac{\left( x_j - \mu_{2_{j,y}} \right)^2}{2\sigma^2} \right) &\geq 0\\
\end{aligned}
\end{equation*}
\begin{equation*}
\begin{aligned}
\ln \left( \frac{P(y=1)}{P(y=0)}\right) + \displaystyle \sum_{j=0}^{d} \left(  \frac{-x_j^2 + 2x_j\mu_{1_{j,y}} - \mu_{1_{j,y}}^2 + x_j^2 - 2x_j\mu_{2_{j,y}} + \mu_{2_{j,y}}^2 }{2\sigma^2} \right) &\geq 0\\
\ln \left( \frac{P(y=1)}{P(y=0)}\right) + \displaystyle \sum_{j=0}^{d} \left(  \frac{\mu_{2_{j,y}}^2  - \mu_{1_{j,y}}^2 + 2(\mu_{1_{j,y}} - \mu_{2_{j,y}})x_j  }{2\sigma^2} \right) &\geq 0\\
b + \displaystyle \sum_{j=0}^{d} x_j w_j &\geq 0
\end{aligned}
\end{equation*}

where 
\begin{equation*}
\begin{aligned}
b=\ln \left( \frac{P(y=1)}{P(y=0)}\right) + \displaystyle \sum_{j=0}^{d} \left(  \frac{\mu_{2_{j,y}}^2  - \mu_{1_{j,y}}^2 }{2\sigma^2} \right) 
\end{aligned}
\end{equation*}

and

\begin{equation*}
\begin{aligned}
w_j = \displaystyle \sum_{j=0}^{d} \frac{\mu_{1_{j,y}} - \mu_{2_{j,y}}}{\sigma^2}
\end{aligned}
\end{equation*}

This means that the classifier has a linear decision boundary.
%%% Local Variables:
%%% mode: latex
%%% TeX-master: "hw6"
%%% End:
